\begin{abstract}

The growing demand on video streaming services increasingly motivates the development of reliable and accurate models for the assessment of the user's Quality of Experience (QoE) in real-time to deliver high-quality streaming content to the user.
However, the complexity caused by the temporal dependencies in sequential QoE data and the non-linear relationships among QoE influence factors has introduced challenges to continuous QoE prediction.
This thesis proposes three novels QoE prediction models in order to improve the QoE prediction performance in terms of prediction accuracy and computational complexity.


\begin{itemize}
  \item First, to enhance the QoE prediction accuracy, the BiLSTM-QoE model is proposed.
  The model utilizes a Bidirectional Long Short Term-Memory network for predicting the user's QoE.  
  
  \item Second, the CNN-QoE model is introduced.
  The model leverages advantages of the Convolutional Neural Network to overcome the computational complexity drawbacks of Long Short Term-Memory networks while improving QoE prediction accuracy.
  Based on a comprehensive evaluation, the CNN-QoE model provides a high QoE prediction performance and outperforms the existing approaches.
  
  \item Third, human-related factors have a significant influence on QoE and play a crucial role in QoE modeling.
  However, these factors were not considered in the BiLSTM-QoE and CNN-QoE model due to the lack of data on QoE influence factors.
  Therefore, in order to precisely model the user's QoE, the impact of the human-related factors, namely perceptual factors, memory effect, and the degree of interest is investigated.
  Based on the investigation, a novel QoE model is proposed that effectively incorporates those factors to reflect the user’s cumulative perception.
  Evaluation results indicate that the model performs excellently in predicting cumulative QoE at any moment within a streaming session.


\end{itemize}


\end{abstract}