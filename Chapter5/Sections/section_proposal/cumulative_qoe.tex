Through the investigation of the above human-related influence factors, the proposed cumulative QoE model is generally presented in Eq.\,\ref{eqn:Cumulative_QoE}. In this model,
to quantify how each of the user's past experiences influences the cumulative perception,
the instantaneous QoE needs to be weighted by the memory effect from the beginning of playback to the investigated time point $t$ within a streaming session.
According to our proposed model, the procedure of estimating cumulative QoE is described as follows: Firstly, the instantaneous QoE is predicted by LSTM-QoE model \cite{QoEModel_LSTM}, and stored into vector ${Q}_{t} = (q_{0},  q_{1}, \dots, q_{t})$. Secondly, the memory weight is calculated by the Eq.\,\ref{eqn:Weight} to form vector ${W}_{t} = (w_{0},  w_{1}, \dots, w_{t})$.

\begin{equation} \label{eqn:Cumulative_QoE}
  CQ_{t} = \lambda_{1} \left ( {Q}_{t}\times{W}^{T}_{t} \right ) + \lambda_{2}DoI
\end{equation}

where $\lambda_{1}$, $\lambda_{2}$ are correlation coefficients which respectively determine the contribution of the user's past experience and user's interest in video content to the predicted cumulative QoE $CQ_{t}$ at time instant $t$.
