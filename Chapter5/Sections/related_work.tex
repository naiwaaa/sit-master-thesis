\subsection{Related Work}

Modeling and predicting the user's cumulative perception to a streaming video content are able to provide lots of advantages for QoE monitor and control systems (e.g., \cite{QoEMonitor1}, \cite{QoEMonitor3}, etc.) since it not only reflects the user's overall satisfaction but also reveals the impact of distorted events happening during the streaming session. Most of existing works only focused on modeling the overall or the instantaneous QoE, which have shown insufficient characteristics. The overall QoE \cite{QoEModel_OP_EstimatingQoE, QoEModel_OP_DerivingValidatingUserExperience, QoEModel_OP_VideoQualityMetric, QoEMonitor2}, which demonstrates the final subjective judgment for a streaming session, can only be assessed when the viewer finishes watching. Therefore, the overall QoE cannot be applied for real-time QoE monitor and also does not give sufficient information about events occurring during the session. Although the instantaneous QoE \cite{QoEModel_IP_TimeVaryingSubjectiveQuality, QoeModel_IP_ShortLongTermQualityModel, QoEModel_BiLSTM}, on the other hand, can provide the instant perceived video quality at a certain moment, it only reflects locally the quality assessment within a specific time range, without considering the cumulative effects of prior events. Hence, it is highly sensitive to video impairments due to hysteresis effect \cite{TemporalHysteresisModel, StallingEvents} and does not precisely express the user's perceived video quality. In contrast, the cumulative QoE effectively leverages the advantages of both the overall and instantaneous QoE, while also eliminating their disadvantages.
 
For those reasons, the idea of cumulative assessment needs to be considered. In \cite{CumulativeQuaityModel}, the cumulative perceptual quality was assessed by leveraging the concept of a sliding window. The work in \cite{CumulativeQoE_Assessing} evaluated the cumulative QoE driven by video quality, bitrate switching, and rebuffering events aligning with the exponential decay of human memory. However, these existing models did not fully express the effects of human-related influence factors (i.e., perceptual factors, primacy, recency, forgetting and repetition, and the user's interest on video content) on the video quality assessment. 

For modeling QoE, there are recent researches working on cyber-physical social systems \cite{QoEModel_Museum1, QoEModel_Museum2}
that capture the human-related factors such as user's profiles, characteristics and interests in order to understand, predict and optimize the user's QoE. However, in the video streaming field, the impacts of human-related factors on QoE are not yet fully considered. A number of existing studies (e.g., \cite{QoEModel_OP_EstimatingQoE, QoEModel_OP_DerivingValidatingUserExperience, QoEModel_OP_VideoQualityMetric, QoEModel_IP_TimeVaryingSubjectiveQuality, QoeModel_IP_ShortLongTermQualityModel}) have considered the perceptual factors (e.g., visual quality, rebuffering events and quality variations) without taking into account the influence of memory on subjective judgment.
To support the idea of utilizing the memory effect on QoE assessment, \cite{NetflixQoE, StallingEvents, LFOVIA} found that primacy and recency effects, which are related to the beginning and the end of a session, respectively, have significant impacts on viewer's perception. These memory effects have also been studied in \cite{QoEModel_OM_BitrateDistribution, QoEModel_OM_DistortionsRebuffMemory, QoEModel_IM_NARX_DynamicNetworks, LFOVIA, QoEModel_IM_TVQoE_ContinuousTimeQoE, QoEModel_IM_NLSS, QoEModel_IM_LSTM_QoE}, resulting in superior performances in terms of accuracy.
In addition, \cite{NetflixQoE,StallingEvents} also indicated that those events happening in the middle of the streaming session also influence the perceived video quality.
Particularly, the user tends to forget the infrequent events, but to remember the repeated one. These memory characteristics actually refer to the forgetting curve \cite{UserForgetful,Ebbinghaus_ForgettingCurve} and repetition \cite{Ebbinghaus_ForgettingCurve}. The work in \cite{CumulativeQoE_Assessing} considered the forgetting behavior in their cumulative QoE model, but did not employ the effect of primacy and recency. On the contrary, the authors in \cite{CumulativeQuaityModel} only investigated the primacy and recency effect. Besides the above influence factors, the effect of video content was also concerned by contemporary works \cite{QoSImpactUserVideoClips,SubjectiveQualityPairedComparison,QoEEvaluation_IPTV_Services,QoEModel_OM_BitrateDistribution,QoeModel_CI_ExpectationConfirmationTheory,QoEModel_IM_TVQoE_ContinuousTimeQoE}. Accordingly, the video content (especially spatial and temporal information) plays an important role in QoE assessments. On the other hand, while \cite{QoEEvaluation_IPTV_Services} indicated that content type has a strong influence, \cite{EyeTracking} explored the user's satisfaction with the quality of a multimedia presentation and user's ability to analyze, synthesize and assimilate the informational content of multimedia. However, to the best of our knowledge, there is no QoE model which takes into account the user's degree of interest in video content. 

In this paper, we propose a QoE model of a cumulative experience driven by human-related factors including perceptual factors, memory effect (primacy, recency and forgetting and repetition) and degree of interest. 