To investigate the trend of user's cumulative QoE at different time instances during a streaming session and compare with our proposed model, we conducted a subjective experiment. In this experiment, 6 distorted videos from the testing set (pattern \#0, \#1, \#3, \#4, \#5, \#7) were selected. Each distorted videos is cropped into 4 small videos with starting timestamps of 00:00:00 and different length (60, 80, 100, and 120 seconds) using FFmpeg \cite{FFmpeg}.

There are totally 100 subjects taking part in this experiment. The Absolute Category Rating method was used i our experiment \cite{ITUT_P913}. The subjects give a rating score at the end of each video with the score ranging from 1 (worst) to 5 (best). Before doing actual subjective tests, the subjects are trained to get accustomed to the rating procedure and the range of video quality. The test sequences are randomly displayed on a 15-inch screen with a resolution of 1920x1080 and a black background. Finally, a screening analysis of the subjective test results is performed and no subject is rejected. The Mean Opinion Score (MOS) is determined as the average of subjects' scores.

For each video, we generated totally 4 small videos, which. The generated videos are divided into 6 sequences with different video content.