The BiLSTM-QoE model achieved high accuracy since it is capable of capturing temporal dependencies in sequential QoE data.
However, the chain structure in LSTM layers requires a high computational cost for practically predicting the user's QoE due to the use of sequential processing over time.
It means that the subsequent processing steps must wait for the output from the previous ones.
This leads to an open question about the performance of the model on real-time QoE-based monitoring and management.

Recently, Temporal Convolutional Network (TCN) \cite{Network_TCN}, a variation of Convolutional Neural Network (CNN), has emerged as a promisingly alternative solution for the sequence modeling tasks.
TCN adopts dilated causal convolutions \cite{Network_Dilated1, Network_Dilated2, Network_Dilated3} to provide a powerful way of extracting the temporal dependencies in the sequential data.
Different from LSTM, the computations in TCN can be performed in parallel, providing computational and modeling advantages.
In practical deployments, TCN convincingly outperforms canonical recurrent architectures including LSTMs and BiLSTMs across a broad range of sequence modeling tasks \cite{Network_TCN}.
Enlightened by the great ability of TCN, an improved TCN-based model, namely QoE-CNN, is introduced for improving the QoE prediction accuracy and optimizing the computational complexity.

The remainder of this chapter is organized as follows:
Section \ref{CNN:sec:TCN} discusses the TCN architecture in detail.
The proposed model is presented in Section \ref{CNN:sec:Proposals}.
Section \ref{CNN:sec:Evaluation} and \ref{CNN:sec:Discussion} provide evaluation results of the proposed model and their discussion, respectively.
Finally, this chapter is concluded in Section \ref{CNN:sec:Summary}.