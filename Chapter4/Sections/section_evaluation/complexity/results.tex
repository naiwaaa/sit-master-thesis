\begin{table}[tb]
  \caption{Computational complexity of the CNN-QoE on the personal computer.}
  \label{tbl:Complexity_PC}
  \centering
    \begin{tabular}{|l|c|c|c|c|}
    \hline
    \multicolumn{1}{|c|}{} & \begin{tabular}[c]{@{}c@{}}Inference Time\\ (ms)\end{tabular} & \begin{tabular}[c]{@{}c@{}}Model Size\\ (kB)\end{tabular} & FLOPs   & \begin{tabular}[c]{@{}c@{}}Number of\\ Parameters\end{tabular} \\ \hline
    CNN-QoE      & \textbf{0.673} &  82.08 & 175,498 &  9,605 \\ \hline
    TCN-QoE   & 0.856 & 145.86 & 253,076 & 13,766 \\ \hline
    LSTM-QoE \cite{QoEModel_LSTM}    & 1.996 &  \textbf{41.18} &  \textbf{30,953} &  \textbf{6,364} \\ \hline
  \end{tabular}

\end{table}


Table \ref{tbl:Complexity_PC} show the computational complexity results of the proposed CNN-QoE compared to the TCN-QoE and LSTM-QoE.
In general, the CNN-QoE requires a higher number of parameters and FLOPs in comparison with LSTM-QoE to achieve higher accuracy.
Although the FLOPs of the proposed model are larger, the inference time is 3 times faster than the LSTM-QoE model.
This indicates that the proposed model can efficiently leverage the power of parallel computation to boost up the computing speed.
It can be seen from Table \ref{tbl:Complexity_PC} that the architecture complexity of TCN-QoE is extremely higher than our proposed CNN-QoE model in terms of number of parameters and FLOPs.
However, the accuracy of TCN-QoE is not quite comparable with the CNN-QoE as shown in Tables \ref{tbl:Accuracy_LFOVIA}, \ref{tbl:Accuracy_LiveMobileStall}, and \ref{tbl:Accuracy_LiveNetflix} .
It proves that the proposed improvement adapted on the original TCN architecture allow CNN-QoE to effectively capture the complex temporal dependencies in a sequential QoE data.


