\Figure[t!][width=\linewidth]{\FigsDir/tho7.png}
   {QoE prediction performance of the CNN-QoE over the LFOVIA Video QoE Database.\label{fig:ResultLfovia}}

\Figure[t!][width=\linewidth]{\FigsDir/tho8.png}
   {QoE prediction performance of the CNN-QoE over the LIVE Mobile Stall II Video Database.\label{fig:ResultLive}}

\Figure[t!][width=\linewidth]{\FigsDir/tho9.png}
   {QoE prediction performance of the CNN-QoE over the LIVE Netflix Video QoE Database.\label{fig:ResultNetflix}}


\begin{table}[t]
  \caption{QoE prediction performance of the CNN-QoE over the LFOVIA Video QoE Database.}
  \label{tbl:Accuracy_LFOVIA}
  \centering
  % min idx: [35,  3, 30, 28, 22, 13, 29, 27]


  \begin{tabular}{|l|c|c|c|}
    \hline
    & PCC & SROCC & RMSE (\%)\\ \hline
    CNN-QoE & \textbf{0.820} & \textbf{0.759} & \textbf{4.81} \\ \hline
    TCN-QoE       & 0.670 & 0.732 & 5.47 \\ \hline
    LSTM-QoE \cite{QoEModel_LSTM}        & 0.800 & 0.730 & 9.56 \\ \hline
    NLSS-QoE \cite{QoEModel_NLSS}        & 0.767 & 0.685 & 7.59 \\ \hline
    SVR-QoE \cite{LFOVIA}         & 0.686 & 0.648 & 10.44 \\ \hline
  \end{tabular}

\end{table}

\begin{table}[t]
  \caption{QoE prediction performance of the CNN-QoE over the LIVE Mobile Stall Video Database II. Boldface indicates the best result.}
  \label{tbl:Accuracy_LiveMobileStall}
  \centering
  % min idx: [ 88,  34,  65, 121,  86, 110,  99,  94]

  \begin{tabular}{|l|c|c|c|}
    \hline
    & PCC & SROCC & RMSE (\%)\\ \hline
    CNN-QoE & \textbf{0.892} & \textbf{0.885} & \textbf{5.36} \\ \hline
    TCN-QoE       & 0.667 & 0.603 & 9.71 \\ \hline
    LSTM-QoE \cite{QoEModel_LSTM}        & 0.878 & 0.862 & 7.08 \\ \hline
    NLSS-QoE    \cite{QoEModel_NLSS}    & 0.680 & 0.590 & 9.52 \\ \hline
  \end{tabular}
\end{table}

\begin{table}[t]
  \caption{QoE prediction performance of the CNN-QoE over the LIVE Netflix Video QoE Database. Boldface indicates the best result.}
  \label{tbl:Accuracy_LiveNetflix}
  \centering
  % min idx: [ 88,  34,  65, 121,  86, 110,  99,  94]

  \begin{tabular}{|l|c|c|c|}
    \hline
    & PCC & SROCC & RMSE (\%)\\ \hline
    CNN-QoE & \textbf{0.848} & \textbf{0.733} & \textbf{6.97} \\ \hline
    TCN-QoE       & 0.753 & 0.720 & 7.62 \\ \hline
    LSTM-QoE \cite{QoEModel_LSTM}       & 0.802 & 0.714 & 7.78 \\ \hline
    NLSS-QoE \cite{QoEModel_NLSS}        & 0.655 & 0.483 & 16.09 \\ \hline
    NARX \cite{QoEModel_NARX_DynamicNetworks}           & 0.621 & 0.557 & 8.52\\ \hline
    \cite{QoEModel_Wireless} & 0.611 & 0.515 & 18.96 \\ \hline
  \end{tabular}
\end{table}


Figures \ref{fig:ResultLfovia}, \ref{fig:ResultLive} and \ref{fig:ResultNetflix} illustrate the QoE prediction performance over the three databases using the proposed CNN-QoE model.
In general, the proposed model produces superior and consistent QoE prediction performance in different situations with and without rebuffering events.
Patterns \#1-\#3 in Figure \ref{fig:ResultLfovia}, \ref{fig:ResultLive}, and \ref{fig:ResultNetflix} show that the proposed model can effectively capture the effect of rebuffering events on the user's subjective QoE.
Especially, even the rebuffering event repeatedly occurs as illustrated in pattern \#3 in Figure \ref{fig:ResultLfovia} and patterns \#2, \#3 in Figure \ref{fig:ResultLive}, the QoE predictions still correlate well with the subjective QoE.
Meanwhile, pattern \#0 in Figure \ref{fig:ResultLfovia} and pattern \#1 in Figure \ref{fig:ResultNetflix} show some fluctuations in the predicted QoE.
However, the amplitudes of these fluctuations are small and the varying trends in the subjective QoE are still adequately captured by the proposed model.
Additionally, a linear trend is subsequently introduced in the predicted QoE after each rebuffering event as shown in patterns \#1 - \#3, Figure 7.
It means that the model is not overfitting and can be trained on the LFOVIA Video QoE Database with larger epochs to further increase its nonlinearity and QoE prediction accuracy.

The QoE prediction performance results over each database in comparison with existing models are shown in the Tables \ref{tbl:Accuracy_LFOVIA}, \ref{tbl:Accuracy_LiveMobileStall} and \ref{tbl:Accuracy_LiveNetflix}.
It is important to note that the Hammerstein-Wiener model in \cite{QoEModel_Wireless} was employed as reported in their work in order to ensure a fair comparison. 
From these tables, it is revealed that the CNN-QoE outperforms the existing QoE models within all the criteria, especially in terms of RMSE.
Moreover, the accuracy produced by CNN-QoE is consistent across the databases, thus marking it as an efficient comprehensive model.
The results illustrate that the CNN-QoE architecture is capable of capturing the complex inter-dependencies and non-linear relationships among QoE influence factors.
Interestingly, there is a significant improvement in QoE prediction accuracy when comparing with TCN-QoE.
It means that the enhancements in the proposed architecture have made the model more suitable for QoE prediction.