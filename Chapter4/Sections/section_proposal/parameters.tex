\begin{table}[t]
  \caption{Hyperparameters for the best performance model.}
  \label{tbl:ModelHyperparameters}
  \centering
  \begin{tabular}{|p{3cm}|c|c|}
  \hline
  Architecture \par Hyperparameters & Description & Derived Value \\
  \hline
  
  $r$ & Receptive field size & 8\\ \hline
  $k$ & Filter size & 2\\ \hline
  $L$ & Number of dilated causal convolution layers & 3\\ \hline
  $n$ & Number of filters on each convolution layer & 32\\ \hline

\end{tabular}

\end{table}

When training the model, an adequate set of architecture hyperparameters must be selected to achieve the best performance.
The proposed model consists of $L$ residual block layers, each layer contains a dilated causal convolution with a filter size of $1 \times k$, as shown in Figure \ref{fig:ProposedArchitecture} and \ref{fig:ProposedResidualBlock}.
Each dilated convolution layer has a dilation factor $d$ doubled at each layer up, as shown in (\ref{eqn:TCN_dilatedFactor}).
The proposed model depends on the network depth $L$ and the filter size $k$.
These hyperparameters control the trade-off between QoE prediction accuracy and computational complexity of the model.
To effectively optimize the hyperparameters, it is important to set a boundary for the space of possible hyperparameter values.

The user's QoE is mostly affected by the recent experiences, also known as the recency effect \cite{Recency, NetflixQoE, LFOVIA}.
The recency effect gradually decreases within 15 to 20 seconds \cite{NetflixQoE, LFOVIA} after distorted events (e.g., bitrate fluctuations, rebuffering events).
Therefore, the effective history or the receptive field size $r$ of the model cannot be larger than 20 time-steps

\begin{equation}
  r \leq 20
  \label{eqn:TCN_receptiveField_limit}
\end{equation}

Moreover, the receptive field depends on the number of dilated causal convolution layers $L$ and the filter size $k$.
For example, with $l \in [1, L]$, the receptive field $r$ can be determined by (\ref{eqn:TCN_receptiveField_2}) \cite{Network_Wavenet, Network_TCN}

\begin{equation}
 r = 2^{L} \mathrm{, if} \quad k = 2
 \label{eqn:TCN_receptiveField_2}
\end{equation}

or (\ref{eqn:TCN_receptiveField_3}) \cite{Network_ReceptiveField_3}

\begin{equation}
  r = 2^{L + 1} - 1 \mathrm{, if} \quad k = 3
  \label{eqn:TCN_receptiveField_3}
\end{equation}

Figure \ref{fig:DilatedConv} shows an example of a three-layer ($L=3$) dilated convolutional network.
In this figure, given the filter size of $1 \times 2$ ($k = 2$), the receptive field is computed by $r = 2^{3} = 8$.
From (\ref{eqn:TCN_receptiveField_limit}), (\ref{eqn:TCN_receptiveField_2}), and (\ref{eqn:TCN_receptiveField_3}), the range of $L$ values can easily be defined $L \in [2, 3, 4]$.

In a 1D convolution, the number of filters $n$ is also important to effectively extract the information from the inputs.
To minimize the computation complexity of the model, the range of $n$ is set to $n \in \{16, 32, 64\}$.
We conduct a simple grid-search of the model architecture hyperparameters with $k \in [2, 3]$, $L \in [2, 3, 4]$, and $n \in \{16, 32, 64\}$.
Table \ref{tbl:ModelHyperparameters} shows the values of $r$, $k$, $L$, and $n$ that achieves the best performance.

% The mean squared error is the default loss function for QoE prediction. Adam is adopted as the optimization strategy, with an initial learning rate set to 0.001.
% The implementations of CNN-QoE are build based on Keras library with the Tensorflow backend.

