In recent years, video streaming has become the most dominant contributor to global Internet traffic.
The Cisco Visual Networking Index forecasts an increase in video traffic, which is expected to reach 82\% by 2021, up from 73\% in 2016 \citep{CiscoIndex}.
The rapid increase of video streaming services creates an extremely huge profit for streaming service providers.
In the context of a highly competitive streaming service market, service providers such as YouTube, Netflix, or Amazon must improve and ensure a sufficient video quality to satisfy the user's expectation, resulting in high quality of experience (QoE).
However, video streaming services are frequently influenced by dynamic network conditions (e.g., throughput, available bandwidth) which can lead to distorted events (e.g., bitrate switching, rebuffering events).
These distorted events can negatively affect the user's satisfaction, resulting in the deterioration of the user's QoE.
The capability of continuously predicting and monitoring the user's QoE in real-time can help video streaming controllers perform a QoE-based network control and management to alleviate the QoE deterioration, resulting in higher overall levels of the user's QoE \citep{QABR_Chanh, QABR_DeepQ}.
Therefore, there is a need for developing reliable QoE prediction models in order to quickly and accurately determine the user's QoE.


However, the continuous prediction of QoE is challenging since the user's QoE is affected by many influence factors such as video quality, video content, bitrate switching, rebuffering, etc.
Moreover, in order to accurately predict the user's QoE, it needs to capture the complex temporal dependencies in sequential QoE data and the non-linear relationships among these QoE influence factors  \citep{EffectSizesOfInfluenceFactors, QoEModel_TimeVaryingSubjectiveQuality, QoEModel_TVQoE_ContinuousTimeQoE, NetflixQoE}.
Additionally, QoE prediction models have to adapt to the dynamic changes in network conditions in real-time.
Therefore, it is necessary to improve the prediction accuracy of QoE models that can perform consistently well across diverse scenarios of video streaming.
Furthermore, it is necessary to optimize the model computational complexity for real-time QoE monitoring.
These factors form the motivation for this thesis.


The main goal of the thesis is to improve the QoE prediction performance in terms of both prediction accuracy and computational complexity of QoE prediction models for QoE-based network control and management.