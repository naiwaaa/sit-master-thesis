QoE modeling for video streaming services has received enormous attention due to its critical importance in QoE-aware applications.
A number of different continuous QoE prediction models have been proposed \cite{QoEModel_TimeVaryingSubjectiveQuality, QoeModel_ShortLongTermQualityModel, QoEModel_BitrateDistribution, QoEModel_NARX_DynamicNetworks, QoEModel_AugmentedAutoregressive, QoEModel_TVQoE_ContinuousTimeQoE, QoEModel_NLSS, QoEModel_3D, QoEModel_Wireless}.
The authors in \cite{QoEModel_TimeVaryingSubjectiveQuality} modeled the time-varying subjective quality (TVSQ) using a Hammerstein-Wiener model.
The work in \cite{QoEModel_AugmentedAutoregressive} proposed a model based on the augmented Nonlinear Autoregressive Network with Exogenous Inputs (NARX) for continuous QoE prediction.
It should be noted that these models did not consider rebuffering events which usually happen in video streaming \cite{Survey_QoE, Survey_QoEModeling}.
On the other hand, the study in \cite{QoEModel_NARX_DynamicNetworks} took into account rebuffering events, perceptual video quality, and memory-related features for QoE prediction.
However, the QoE prediction accuracy varied across different video streaming scenarios.
The reason is that the model suffered from the difficulty in capturing the complex dependencies among QoE influence factors, leading to unreliable and unstable QoE prediction performances.


In order to address the above challenges, the authors in \cite{QoEModel_LSTM} proposed a QoE prediction model, namely, LSTM-QoE, which was based on Long Short-Term Memory networks (LSTM).
The authors argued that the continuous QoE is dynamic and time-varying in response to QoE influencing events such as rebuffering \cite{MeasuringQoEOfHTTP} and bitrate adaptation \cite{QoEModel_TimeVaryingSubjectiveQuality}.
To capture such dynamics, LSTM was employed and the effectiveness in modeling the complex temporal dependencies in sequential QoE data was shown.
The model was evaluated on different QoE databases and outperformed the existing models in terms of QoE prediction accuracy.
However, the computational complexity of the model was not fully inspected.
Since the recurrent structure in LSTM can only process the task sequentially, the model failed to effectively utilize the parallel computing power of modern computers, leading to a high computational cost.
Therefore, the feasibility to utilize the LSTM-QoE model for real-time video streaming applications remains an open question.


Therefore, in this thesis, we aim to improve the QoE prediction performance of QoE prediction models by enhancing the QoE prediction accuracy and optimizing the computational complexity.