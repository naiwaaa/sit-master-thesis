In order to improve and monitor the user's QoE, video streaming service providers have to develop various techniques to measure the QoE.
These techniques assess the user's QoE and quantify it in measurable metrics.
The user's QoE can be assessed both subjectively and objectively.


Subjective QoE assessment involves users and requires user surveys to gather subjective evaluations of a given service.
The most common method to capture subjective QoE evaluations is Mean Opinion Score (MOS) which is an ITU standardized \citep{QoEAss_ITU}.
MOS is a 5-point scale ranging from 1 to 5, which correlates to bad, poor, fair, good, and excellent.
Despite the high accuracy of the subjective assessment, it does have certain disadvantages. Firstly, a large number of participants need to be gathered in order to conduct the surveys. Secondly, this method is generally expensive in terms of cost and time consumption. Finally, it cannot be used for real-time QoE measurement or monitoring for video streaming.


Alternatively, objective QoE assessment is the most frequently used technique since it can be used to estimate the user's QoE without requiring human interaction.
Objective QoE models are more suited for real-time QoE measurements. However, it can be less accurate than subjective QoE evaluations.