\chapter{Discussion}
\label{ch:Discussion}


\renewcommand{\SectionsDir}{Chapter6/Sections}
\renewcommand{\FigsDir}{Chapter6/Figs}
\renewcommand{\TablesDir}{Chapter6/Tables}


This chapter discusses the QoE prediction performance of the three QoE models proposed in this thesis.
Section \ref{Discussion:sec:InstantaneousQoE} discusses the performance of two instantaneous QoE prediction models which are BiLSTM-QoE and CNN-QoE. 
Section \ref{Discussion:sec:CumulativeQoE} summarized the advantages and remaining issues of the cumulative QoE prediction model.


\section{QoE prediction performance of BiLSTM-QoE and CNN-QoE models}
\label{Discussion:sec:InstantaneousQoE}

\Table[tb]{\TablesDir/compare}
  {QoE prediction accuracy of the BiLSTM-QoE and CNN-QoE over the LIVE Netflix Video QoE Database.\label{tbl:Compare}}

In term of accuracy, the BiLSTM-QoE and CNN-QoE models are both outperforms the existing studies, as shown in Section \ref{BiLSTM:sec:Evaluation} and \ref{CNN:sec:Evaluation}.
Table \ref{tbl:Compare} tabulated the comparison in QoE prediction accuracy between our models and reference models over the LIVE Netflix Video QoE Database.
Accordingly, the CNN-QoE model provides a competitive performance in terms of PCC and SROCC against the BiLSTM-QoE model.
It should be noted that the BiLSTM-QoE model was evaluated on only one database.
In contrast, three different databases were used to assess the performance of the CNN-QoE model.
Section \ref{CNN:sec:Evaluation} showed that the CNN-QoE model can perform consistently well across the QoE databases.


Moreover, we also introduce several improvements to the CNN-QoE architecture to overcome the computational complexity drawbacks of LSTM-based QoE models.
These improvements helped the model run faster than the reference models, leading to real-time QoE prediction advantages.
Therefore, the CNN-QoE model can be an excellent choice for predicting the instantaneous QoE.


\section{Cumulative QoE prediction model}
\label{Discussion:sec:CumulativeQoE}

The results of the cumulative QoE prediction model validated the impact of memory effects and DoI on the user's QoE.
Moreover, the model can quickly and precisely estimate the cumulative QoE in the experiment.
However, it still has some limitations.
First, the model relies on an instantaneous QoE prediction model to predict the user's cumulative QoE due to the lack of data on subjective QoE evaluations.
Second, the correlation between DoI and the user's QoE was not so high, hence, the prediction accuracy of the model is perhaps not sufficient.
Finally, the model should be evaluated in multiple databases to understand how well the model will perform across diverse scenarios of video streaming.
