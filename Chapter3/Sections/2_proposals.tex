\Figure[tb][width=0.8\linewidth]{\FigsDir/BiLSTM.png}
  {The proposed BiLSTM-QoE architecture.\label{fig:BiLSTM}}


The work in \citep{QoEModel_LSTM} predicts the user's QoE using LSTM which is a special kind of Recurrent Neural Networks.
With recurrent connections in the hidden layer, LSTM can capture temporal dependencies in sequential QoE data.
However, it is very complex to rely on a single LSTM to capture these dependencies and the complex interactions among QoE influence factors such as video quality, bitrate switching, rebuffering events.
Moreover, unidirectional LSTM structure only considers forward dependencies, thus, useful information may be filtered out.
Therefore, a novel QoE prediction model based on BiLSTM is taken into account.


BiLSTM processes sequential features in both forward and backward directions, thereby it can access temporal dependencies in both directions and result higher level of representations of input features.
The bidirectional networks have been proved that it has better performance compare with unidirectional ones in many fields \citep{BiLSTM_TrafficSpeed, BiLSTM_HybridSpeech, BiLSTM_NonNativeSpeech}.


The proposed architecture of the BiLSTM-QoE model, illustrated in Figure \ref{fig:BiLSTM}, consists of two main layers: BiLSTM layer and LSTM layer.
BiLSTM layer connects two separate hidden layers to learn the sequential input features in two directions (forward and backward).
The output of BiLSTM layer will be fed into the LSTM layer to predict the instantaneous QoE.